%% For double-blind review submission, w/o CCS and ACM Reference (max submission space)
\documentclass[sigplan,10pt,review]{acmart} %,anonymous]{acmart}
\settopmatter{printfolios=true,printccs=false,printacmref=false}
%% For double-blind review submission, w/ CCS and ACM Reference
%\documentclass[sigplan,review,anonymous]{acmart}\settopmatter{printfolios=true}
%% For single-blind review submission, w/o CCS and ACM Reference (max submission space)
%\documentclass[sigplan,review]{acmart}\settopmatter{printfolios=true,printccs=false,printacmref=false}
%% For single-blind review submission, w/ CCS and ACM Reference
%\documentclass[sigplan,review]{acmart}\settopmatter{printfolios=true}
%% For final camera-ready submission, w/ required CCS and ACM Reference
%\documentclass[sigplan]{acmart}\settopmatter{}

\usepackage{xcolor}
\usepackage{listings}
\usepackage{float}
\usepackage{graphicx}
\graphicspath{ {./} }

\floatstyle{boxed}
\restylefloat{figure}
%% Conference information
%% Supplied to authors by publisher for camera-ready submission;
%% use defaults for review submission.
\acmConference[CMPUT500]{Foundations of Program Analysis}{Fall 2019}{}
\acmYear{2019}
\acmISBN{} % \acmISBN{978-x-xxxx-xxxx-x/YY/MM}
\acmDOI{} % \acmDOI{10.1145/nnnnnnn.nnnnnnn}
\startPage{1}

%% Copyright information
%% Supplied to authors (based on authors' rights management selection;
%% see authors.acm.org) by publisher for camera-ready submission;
%% use 'none' for review submission.
\setcopyright{none}
%\setcopyright{acmcopyright}
%\setcopyright{acmlicensed}
%\setcopyright{rightsretained}
%\copyrightyear{2018}           %% If different from \acmYear

%% Bibliography style
\bibliographystyle{ACM-Reference-Format}
%% Citation style
%\citestyle{acmauthoryear}  %% For author/year citations
%\citestyle{acmnumeric}     %% For numeric citations
%\setcitestyle{nosort}      %% With 'acmnumeric', to disable automatic
                            %% sorting of references within a single citation;
                            %% e.g., \cite{Smith99,Carpenter05,Baker12}
                            %% rendered as [14,5,2] rather than [2,5,14].
%\setcitesyle{nocompress}   %% With 'acmnumeric', to disable automatic
                            %% compression of sequential references within a
                            %% single citation;
                            %% e.g., \cite{Baker12,Baker14,Baker16}
                            %% rendered as [2,3,4] rather than [2-4].


%%%%%%%%%%%%%%%%%%%%%%%%%%%%%%%%%%%%%%%%%%%%%%%%%%%%%%%%%%%%%%%%%%%%%%
%% Note: Authors migrating a paper from traditional SIGPLAN
%% proceedings format to PACMPL format must update the
%% '\documentclass' and topmatter commands above; see
%% 'acmart-pacmpl-template.tex'.
%%%%%%%%%%%%%%%%%%%%%%%%%%%%%%%%%%%%%%%%%%%%%%%%%%%%%%%%%%%%%%%%%%%%%%


%% Some recommended packages.
\usepackage{booktabs}   %% For formal tables:
                        %% http://ctan.org/pkg/booktabs
\usepackage{subcaption} %% For complex figures with subfigures/subcaptions
                        %% http://ctan.org/pkg/subcaption
\usepackage{hyperref}   %% For clickable links.

\hypersetup{
    colorlinks=true,
    linkcolor=blue,
    filecolor=magenta,
    urlcolor=cyan,
}



\begin{document}

%% Title information
\title{Compiling Rust at Runtime}         %% [Short Title] is optional;
                                        %% when present, will be used in
                                        %% header instead of Full Title.
%\titlenote{with title note}             %% \titlenote is optional;
                                        %% can be repeated if necessary;
                                        %% contents suppressed with 'anonymous'
%\subtitle{Subtitle}                     %% \subtitle is optional
%\subtitlenote{with subtitle note}       %% \subtitlenote is optional;
                                        %% can be repeated if necessary;
                                        %% contents suppressed with 'anonymous'


%% Author information
%% Contents and number of authors suppressed with 'anonymous'.
%% Each author should be introduced by \author, followed by
%% \authornote (optional), \orcid (optional), \affiliation, and
%% \email.
%% An author may have multiple affiliations and/or emails; repeat the
%% appropriate command.
%% Many elements are not rendered, but should be provided for metadata
%% extraction tools.

%% Author with single affiliation.
\author{Batyr Nuryyev}
%\authornote{with author1 note}          %% \authornote is optional;
                                        %% can be repeated if necessary
\orcid{nnnn-nnnn-nnnn-nnnn}             %% \orcid is optional
\affiliation{
  %\position{Graduate Student}
  \department{Department of Computing Science}              %% \department is recommended
  \institution{University of Alberta}            %% \institution is required
  % \streetaddress{Street1 Address1}
  % \city{City1}
  % \state{State1}
  % \postcode{Post-Code1}
  % \country{Country1}                    %% \country is recommended
}
\email{nuryyev@ualberta.ca}          %% \email is recommended

%% Author with two affiliations and emails.
% \author{First2 Last2}
% \authornote{with author2 note}          %% \authornote is optional;
%                                         %% can be repeated if necessary
% \orcid{nnnn-nnnn-nnnn-nnnn}             %% \orcid is optional
% \affiliation{
%   \position{Position2a}
%   \department{Department2a}             %% \department is recommended
%   \institution{Institution2a}           %% \institution is required
%   \streetaddress{Street2a Address2a}
%   \city{City2a}
%   \state{State2a}
%   \postcode{Post-Code2a}
%   \country{Country2a}                   %% \country is recommended
% }
% \email{first2.last2@inst2a.com}         %% \email is recommended
% \affiliation{
%   \position{Position2b}
%   \department{Department2b}             %% \department is recommended
%   \institution{Institution2b}           %% \institution is required
%   \streetaddress{Street3b Address2b}
%   \city{City2b}
%   \state{State2b}
%   \postcode{Post-Code2b}
%   \country{Country2b}                   %% \country is recommended
% }
% \email{first2.last2@inst2b.org}         %% \email is recommended


%% Abstract
%% Note: \begin{abstract}...\end{abstract} environment must come
%% before \maketitle command
\begin{abstract}

Rust is statically-compiled programming language
that focuses on memory and and type safety.
The safety is ensured at static time through


\end{abstract}


%% 2012 ACM Computing Classification System (CSS) concepts
%% Generate at 'http://dl.acm.org/ccs/ccs.cfm'.
% \begin{CCSXML}
% <ccs2012>
% <concept>
% <concept_id>10011007.10011006.10011008</concept_id>
% <concept_desc>Software and its engineering~General programming languages</concept_desc>
% <concept_significance>500</concept_significance>
% </concept>
% <concept>
% <concept_id>10003456.10003457.10003521.10003525</concept_id>
% <concept_desc>Social and professional topics~History of programming languages</concept_desc>
% <concept_significance>300</concept_significance>
% </concept>
% </ccs2012>
% \end{CCSXML}
%
% \ccsdesc[500]{Software and its engineering~General programming languages}
% \ccsdesc[300]{Social and professional topics~History of programming languages}
%% End of generated code


%% Keywords
%% comma separated list
\keywords{compilers, rust, static-analysis}  %% \keywords are mandatory in final camera-ready submission


%% \maketitle
%% Note: \maketitle command must come after title commands, author
%% commands, abstract environment, Computing Classification System
%% environment and commands, and keywords command.
\maketitle


\section{Introduction}

% Here I will explain that currently Rust compiler is very slow
% due to its static checks. What if we could keep static checks,
% but rather compile just-in-time to achieve speed, etc.
% And focus JIT on hot spot places in the program, etc.



\subsection{Static vs. Runtime Execution Trade-Off}
Rust is a multi-paradigm systems programming language focused on type and
memory safety.  It is syntactically similar to C++. Unlike C++, it has a
stricter type system as well as guarantees compile-time memory safety. Since
the Rust compiler enforces these guarantees at static time, it attains high
execution performance at run time. It also allows for ``fearless concurrency'',
where most of the bugs stemming from the use of concurrency concepts (e.g.,
threads) are detected at compile time. Being able to detect concurrency-related
bugs before execution helps Rust developers trace bugs easier and fix 
compile-time errors faster.

% https://benchmarksgame-team.pages.debian.net/benchmarksgame/fastest/gcc-rust.html

However, while the Rust compiler, \texttt{rustc}, performs these checks at
compile time, its compilation time drastically increases. According to the
\textit{Computer Language Benchmarks Game}, Rust performed about slower on 6
out of 10 programs when comparing Rust to C \cite{rustbench}. On these 6
programs, the \texttt{rustc} was 11\% slower on average than GCC.

Therefore, slow compile times open up opportunities for the ``just-in-time''
(or dynamic) compilation techniques that may speed up static compilation
without hurting execution speed.

\subsection{Potential solution}


One way to alleviate the problem of slow compilation would be to compile
Rust at runtime. Compiling code at run-time encompasses, in general, two forms
of execution: \textit{interpretation}, which immediately executes a block of code;
and \textit{just-in-time (JIT) compilation}, which first compiles a block of code
(usually a method or function) to native machine code, and then executes it.
Languages that are compiled at runtime usually require an environment
that includes both interpreter and JIT compiler. Such environment is
called a (process) \textit{virtual machine} (VM); it manages
the compilation process and decides whether to interpret or JIT-compile
a block of code. The decision depends on the character of the block of code; 
for example, if the block of code is to known to be
frequently called throughout a lifetime of a program, it would be more
optimal to JIT-compile it and cache the result for future calls; but,
if the code is rarely called and is trivial to execute (i.e., does not
contain complex, resource-intensive operations to optimize), interpreting it
would be faster.

Rust could leverage run-time techniques to speed up its compilation time.
Knowing that very few parts in the code get executed the most, one could
focus on optimizing those parts through JIT-compilation, and interpreting the
rest.

Compiling language at runtime is not a novel idea.  Inspired by Smalltalk, Java
is a strongly-typed static object-oriented programming language.  Its execution
model (Java Virtual Machine, or JVM) consists of two compilers \cite{javajvm}.
The front-end compiler performs necessary static checks (e.g., type checking)
at compile time, and emits an intermediate representation, or
\textit{bytecode}.  Then, the other compiler takes the bytecode, performs
necessary optimizations, and JIT-compiles it (by translating it into machine
code and executing it). JVM attempts to optimize more of so-called \textit{hot
spots}, the areas of code where a program spends most of its time. As Java is
similar to Rust in its type system (strongly-typed and static), similar dynamic
compilation techniques will help us implement similar runtime environment for
Rust.

Before discussing core ideas and implementation of our runtime
execution model, we will briefly describe Rust itself, and the features
that make it unique among other languages.


\section{Overview of Rust lang}

% - Here I will explain what Rust is, how it looks like.
% - Borrow checker, how it works. Example with lifetimes. Ref: Rustbook
% - Talk about Cargo and Rust compiler - outputs LLVM IR, which then
%   generates machine code, etc.

\begin{figure*}[ht]
    \begin{center}
    \lstinputlisting{rust_ex1.rs}
    \end{center}
    \caption{Snippet of Rust lang \textcolor{red}{cite the book}}
    \label{rustex1}
\end{figure*}

As stated in the previous section, Rust is a strongly typed language compiled
at static time.  Each variable should have its type explicitly declared (or
easily inferable) at compile time. However, type system alone is not enough to
make the language more memory secure. Thus, a concept of \textit{ownership},
which I will explain in this section, is what makes Rust truly a unique
language.

Before understanding ownership, let us consider how programs manage memory.
Some languages use a \textit{garbage collector}; while program is running,
it looks for unused memory locations and frees them constantly (i.e., at
run time). Other languages completely delegate memory management to developers
themselves; for example, if you are writing in C and want to allocate a variable at
the heap, you have to allocate and free memory on your own using \texttt{malloc}.

Garbage collectors usually add runtime overhead because the collector has to find
all (and there can be many at a time) allocated, but unused memory segments,
and deallocate them. On the other hand, trusting developers to manage program
memory manually is a \textit{bad} and error-prone idea that may lead to problems such
as dangling pointer (a pointer that points to deallocated memory
location), memory leaks (forgetting to deallocate allocated memory), and others which
are hard to detect and debug.

However, Rust introduces a third dimension of memory management
that is enforced at compile time. According to Steve Klabnik and Carol Nichols
\textcolor{red}{cite rust book}, "memory is managed through a \textit{system of
ownership} with a set of rules that the compiler checks at compile time. None of
the ownership features slow down your program while it's running." Note
that even though ownership features do not slow down the execution of a program,
it slows down compilation by having to perform additional analysis
to verify the rules.

The following are the ownership rules that help Rust get rid of garbage collection
or manual memory management:

\begin{itemize}
    \item Each value has its \textit{owner} (a.k.a., a variable).
    \item Each value has \textit{a single owner} at a time.
    \item Value is dropped when an owner goes out of scope.
\end{itemize}

Take a look at \hyperref[rustex1]{Figure 1}. Before the declaration statement,
\texttt{s} is not valid in the nested scope. After it is declared, 
\texttt{s} can now be used, but only in the nested scope. As the scope
is over, \texttt{s}'s value is dropped as owner goes out of the nested scope.

What makes ownership more exciting is so-called \textit{borrowing}. For comparison
with \hyperref[rustex1]{Figure 1}, take a look at \hyperref[rustex2]{Figure 2}.

\begin{figure*}[ht]
    \begin{center}
    \lstinputlisting{rust_ex2.rs}
    \end{center}
    \caption{Borrowing a value with references.}
    \label{rustex2}
\end{figure*}

The \texttt{owner} is of type of struct \texttt{Foo}, which may contain some fields.
Instead of either copying or \textit{moving} the value of that struct into the variable
\texttt{borrower}, we can instead borrow a value of \texttt{owner} by \textit{referencing}
it.

Several questions immediately raise - what about multiple references? If I have 
multiple references that change (or, \textit{mutate}) the value of \texttt{owner},
would not this cause \textit{data races}? Before providing answers, we have to differentiate
between \textit{immutable} and \textit{mutable} references. The immutable references do not
modify the original value (i.e., read-only), and the mutable ones may
change the original value (i.e., both read and write accesses). To borrow
any object in a \textit{mutable} way, you can use \texttt{\&mut} instead of
original \texttt{\&} (immutable by default).

The \textbf{borrow checker}, that statically checks the validity of references (or borrows),
also follows a system of rules. The following are some of its rules:

\begin{itemize}
    \item You can have either one mutable reference, or any number of immutable references,
          but not both.
    \item References must always be valid (i.e., cannot reference values when they are out
         of their scope).
\end{itemize}

Besides borrow checking, there are some other important (for compile-time safety)
analyses that compiler runs; examples include \textit{lifetime analysis}, \textit{typestate
analysis}, and so on. Because borrow checker is the only analyzer I am looking at
in this paper, we are not going to consider the other ones.


\section{Main ideas}

% - Here I will again remark on just-in-time benefits. Cite Java paper here,
%   how it generates code based on bytecodes.
% - Then, talk briefly about ASTs and how we wrote an interpreter that walks
%   the AST tree as a visitor. Cite "crafting interpreters"
% - Talk about missing garbage collection.
% - Talk about separation of concerns, e.g. separating static checks into
%   one part, and compilation and optimization into the other.

This is my main idea!


\section{Implementation}

% - talk about initial challenges as well
% - Talk about front-end: lexer and parser.
% - Talk about interpreter. How visitor patterns helps interpreter traverse a tree
%   in an extensible way.
% - Talk about Eclipse OMR and how you integrate it.

This is implementation!


\section{Evaluation}

% - TODO: We gotta do some evaluation here ... Run 2-3 simple programs about 1000 times.
%   Get their average real compilation and execution speed, together.
% - Talk about simplicity of our compiler. Cannot provide a complete evaluation yet.
% - and give graphs based on point 1.
% - plus talk about standard benchmarks. We do not support all of the grammar yet,
%   but when we do, we will have to run them on the standard benchmark that includes
%   real programs as well as stress tests.

I did not perform any benchmarking or evaluation
program because the dynamic compilation infrastructure
does not support ``enough'' of the language yet.
There are, however, official benchmarks that Rust
compiler is evaluated against. These are the
programs that I plan to verify my main ideas (hypothesis)
against, to see how well a dynamic compilation system works.

As such, there is currently only support for
unary, binary expressions; variable declaration
statements; assignment statement expressions;
as well as references (accessing an address,
or equivalently, dereferencing an element).
The next step would be to introduce a notion of \textit{scope}
(as well as functions) into the program.
After scopes, functions, lifetimes, and (at least) closures
I could perform benchmarking against the set of programs
provided by Rust community. \textcolor{red}{cite benchmark Rust programs
+ state that in presentation, but show example}


\section{Related Work}

% - Talk about related work in the industry. How our work helps ? how is it better than others?
%   How does it fill in the gap?

There is currently no dynamic compiler (whether an interpreter or JIT) that
entirely supports Rust. For example, there is HolyJit, a "generic purpose
just-in-time compiler for Rust" \cite{holyjit}. It extends static Rust
compiler by obtaining its intermediate representation (MIR) and converts it
into a representation for the JIT compiler from HolyJit library. Despite
supporting much more language features than my system, there is currently no
one working actively on the project. As it has been inactive since mid-2018,
not much can be done given that Rust itself evolves quite rapidly (e.g., it has
recently introduced \texttt{async/await} into the language).

Nevertheless, the Rust compiler, \texttt{rustc}, is also getting
faster ``with every commit''. Section 1 briefly desribes
the performance of the default Rust compiler.


\section{Conclusion}

% Talk about Rust's async await and potential need for runtime. Cite how it's implemented,
% and how Rust may consider adding garbage collection.

this is conclusion!!


%% Acknowledgments
% \begin{acks}                            %% acks environment is optional
%                                         %% contents suppressed with 'anonymous'
%   %% Commands \grantsponsor{<sponsorID>}{<name>}{<url>} and
%   %% \grantnum[<url>]{<sponsorID>}{<number>} should be used to
%   %% acknowledge financial support and will be used by metadata
%   %% extraction tools.
%   This material is based upon work supported by the
%   \grantsponsor{GS100000001}{National Science
%     Foundation}{http://dx.doi.org/10.13039/100000001} under Grant
%   No.~\grantnum{GS100000001}{nnnnnnn} and Grant
%   No.~\grantnum{GS100000001}{mmmmmmm}.  Any opinions, findings, and
%   conclusions or recommendations expressed in this material are those
%   of the author and do not necessarily reflect the views of the
%   National Science Foundation.
% \end{acks}


%% Bibliography
%\bibliography{bibfile}


%% Appendix
% \appendix
% \section{Appendix}
%
% Text of appendix \ldots

\end{document}
