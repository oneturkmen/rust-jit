In this paper, we took a look at a potential idea of dynamic compilation of
Rust. We saw that the static analysis checks such as borrow checker can be
separated into their own stage in the pipeline. The pipeline, thus, contains
both static and dynamic aspects: on one side, we are validating the source for
ownership (and other) rules; on the other, we are dynamically compiling,
optimizing (in a \textit{smarter} way) and generating machine code at runtime.

In future, one could look into branching off directly from the static Rust
compiler.  That way, we do not have to think about implementing front-end
(performing lexical, syntactic and semantic analysis) because the work has
already been done by the Rust compiler itself. In addition, using the HIR/MIR
representation would give us more semantic information about the source that I
currently generate and store. The same information could be used to perform
additional analyses (e.g., taint analysis).
